%-----------------------------------------------------------------------------------------------------------
% packages
%------------------------------------------------------------------------------------------------------------


\ifx\firstpassage\undefined
\def\firstpassage{1}
\newcount\nbpassages
\nbpassages=1



%-----------------------------------------------------------------------------------------------------------
% packages
%------------------------------------------------------------------------------------------------------------
%\documentclass[french,11pt]{report}
\documentclass{report}
\usepackage[dvips]{graphicx}
\usepackage {epsfig}
\usepackage[french]{babel}
\usepackage{a4}
\usepackage[T1]{fontenc}
\usepackage{amsmath}
\usepackage{amssymb}
\usepackage{subfigure}
\usepackage{float}
\usepackage{latexsym}
\usepackage{amsfonts}
\usepackage{epic}
\usepackage{eepic}
\usepackage{makeidx}
\usepackage{multido}
\usepackage{varindex}

\usepackage{graphicx}   %[pdftex]
\usepackage[margin=0.5cm]{geometry}
%\usepackage{hyperref}  %lien en PDF

\makeatletter
\renewcommand{\thepage}{}


%------------------------------------------------------------------------------------------------------------
% index
%------------------------------------------------------------------------------------------------------------
%\varindexIndex
\newcommand{\indexfr}  [1]{\index{#1@#1}}
\newcommand{\indexfrr} [2]{\index{#1@#1!#2@#2}}
\newcommand{\indexfrrr}[3]{\index{#1@#1!#2@#2!#3@#3}}
\newcommand{\indexsee} [2]{\index{#1@#1|see{#2}}}

    
%------------------------------------------------------------------------------------------------------------
% fonctions
%------------------------------------------------------------------------------------------------------------

\newcommand{\parenthese}[1]{\left(#1\right)}
\newcommand{\pa}[1]{\left(#1\right)}
\newcommand{\bigpa}[1]{\big( #1 \big)}
\newcommand{\biggpa}[1]{\Big( #1 \Big)}
\newcommand{\bigggpa}[1]{\bigg( #1 \bigg)}
\newcommand{\biggggpa}[1]{\Bigg( #1 \Bigg)}

\newcommand{\scal}[1]{\left<#1\right>}

\newcommand{\abs}[1]{\left|#1\right|}
\newcommand{\bigabs}[1]{\big|#1\big|}
\newcommand{\biggabs}[1]{\bigg|#1\bigg|}

\newcommand{\accolade}[1]{\left\{#1\right\}}
\newcommand{\acc}[1]{\left\{#1\right\}}
\newcommand{\bigacc}[1]{\big\{#1\big\}}
\newcommand{\biggacc}[1]{\Big\{#1\Big\}}
\newcommand{\bigggacc}[1]{\bigg\{#1\bigg\}}
\newcommand{\biggggacc}[1]{\Bigg\{#1\Bigg\}}

\newcommand{\crochet}[1]{\left[#1\right]}
\newcommand{\cro}[1]{\left[#1\right]}
\newcommand{\bigcro}[1]{\big[ #1 \big]}
\newcommand{\biggcro}[1]{\Big[ #1 \Big]}
\newcommand{\bigggcro}[1]{\bigg[ #1 \bigg]}
\newcommand{\biggggcro}[1]{\Bigg[ #1 \Bigg]}
\newcommand{\crochetopen}[1]{\left]#1\right[}

\newcommand{\indicatrice}[1]{\mathbf{1}_{\accolade{#1}}}

\newcommand{\ensemblereel}[0]{\mathbb{R}}
\newcommand{\R}[0]{\mathbb{R}}

\newcommand{\ensembleentier}[0]{\mathbb{N}}
\newcommand{\N}[0]{\mathbb{N}}

\newcommand{\infegal}[0]{\leqslant}
\newcommand{\supegal}[0]{\geqslant}

\newcommand{\intervalle}[2]{\left\{#1,\cdots,#2\right\}}
\newcommand{\intervalleinf}[3]{#1 \infegal #2 \infegal #3}

\newcommand{\loinormale}[2]{{\cal N} \parenthese {#1,#2}}
\newcommand{\loibinomiale}[2]{{\cal B} \parenthese {#1,#2}}
\newcommand{\loimultinomiale}[1]{{\cal M} \parenthese {#1}}
\newcommand{\loi}[0]{{\cal L}}

\newcommand{\dans}[0]{\longrightarrow}

\newcommand{\demonstration}[0]{\bigskip\textbf{D�monstration :}\bigskip}

\newcommand{\partialfrac}[2]{\frac{\partial #1}{\partial #2}}

\newcommand{\summy}[2]{\overset{#2}{\underset{#1}{\sum}}}
\newcommand{\bigsummy}[2]{\overset{#2}{\underset{#1}{\displaystyle\sum}}}
\newcommand{\prody}[2]{\overset{#2}{\underset{#1}{\prod}}}
\newcommand{\bigprody}[2]{\overset{#2}{\underset{#1}{\displaystyle\prod}}}
\newcommand{\summyone}[1]{\underset{#1}{\sum}}
\newcommand{\bigsummyone}[1]{\underset{#1}{\displaystyle\sum}}
\newcommand{\prodyone}[1]{\underset{#1}{\prod}}

\newcommand{\card}[1]{card \parenthese{#1}}
\newcommand{\esperance}[1]{\mathbb{E}\parenthese{#1}}
\newcommand{\esperanceseul}[0]{\mathbb{E}}
\newcommand{\esp}[1]{\mathbb{E}\pa{#1}}

\newcommand{\pr}[1]{\mathbb{P}\pa{#1}}

\newcommand{\norme}[1]{\left\Vert#1\right\Vert}

%\newcommand{\independant}[0]{\;\makebox[3ex]{
%					\makebox[0ex]{\rule[-0.2ex]{3ex}{.1ex}}
%					\makebox[.5ex][l]{\rule[-.2ex]{.1ex}{2ex}}
%					\makebox[.5ex][l]{\rule[-.2ex]{.1ex}{2ex}}}}

\newcommand{\independant}[0]{\;\makebox[3ex]{
     					\makebox[0ex]{\rule[-0.2ex]{3ex}{.1ex}}
     \!\!\!\!	\makebox[.5ex][l]{\rule[-.2ex]{.1ex}{2ex}}
         			\makebox[.5ex][l]{\rule[-.2ex]{.1ex}{2ex}}} \,\,}
         
\newcommand{\vecteur}[2]{\pa{#1,\dots,#2}}
\newcommand{\vecteurno}[2]{#1,\dots,#2}

\newcommand{\union}[2]{\overset{#2}{\underset{#1}{\bigcup}}}
\newcommand{\unionone}[1]{\underset{#1}{\bigcup}}

\newcommand{\para}[1]{\bigskip\textbf{#1}}
\newcommand{\QARRAY}[2]{\left \{ \begin{array}{l} #1 \\ #2 \end{array} \right .}
\newcommand{\QARRAYno}[2]{\left . \begin{array}{l} #1 \\ #2 \end{array} \right .}
\newcommand{\sachant}{\; | \;}
\newcommand{\itemm}[0]{\item[\quad -]}
\newcommand{\ensemble}[2]{\acc{#1,\dots,#2}}

\newcommand{\twoindices}[2]{\begin{subarray}{c} #1 \\ #2 \end{subarray}}
\newcommand{\vecteurim}[2]{\pa{\begin{subarray}{c} #1 \\ #2 \end{subarray}}}
\newcommand{\vecteurimm}[2]{\pa{\begin{array}{c} #1 \\ #2 \end{array}}}

\newcommand{\sac}{\sachant}
\newenvironment{eqnarrays} {\begin{eqnarray}}{\end{eqnarray}}

\newcommand{\dia}{-}


%------------------------------------------------------------------------------------------------------------
% fonctions sp�ciales
%------------------------------------------------------------------------------------------------------------

% �crire soi-m�me le num�ro d'une �quation
\newcommand{\numequation}[0]{\stepcounter{equation}(\theequation)}		


%------------------------------------------------------------------------------------------------------------
% dessin
%------------------------------------------------------------------------------------------------------------

\newcounter{putx}
\newcounter{puty}

% dessin d'un arc sup�rieur : x,y,size
\newcommand{\arcup}[3]{
                                \put(#1,#2){\arc{#3}{2.3}{6.8}}
                                \setcounter{putx}{#1}\addtocounter{putx}{6}
                                \setcounter{puty}{#2}\addtocounter{puty}{-4}
                                \put(\value{putx},\value{puty}){\vector(0,-1){1}}
                      }
                      
% dessin d'un arc inf�rieur : x,y,size
\newcommand{\arcdown}[3]{
                                \put(#1,#2){\arc{#3}{5.8}{10.5}}
                                \setcounter{putx}{#1}\addtocounter{putx}{6}
                                \setcounter{puty}{#2}\addtocounter{puty}{3}
                                \put(\value{putx},\value{puty}){\vector(0,1){1}}
                        }
% dessin un rectangle
\newcommand{\drawbox}[4]{\drawline(#1,#2)(#3,#2)(#3,#4)(#1,#4)(#1,#2)}


%------------------------------------------------------------------------------------------------------------
% annexe
%------------------------------------------------------------------------------------------------------------

% voir les annexes : label, marques dans l'index
\newcommand{\seeannex}[2]{\indexfrr{annexes}{#2}
													\footnote{Annexes~: voir paragraphe~\ref{#1}, page~\pageref{#1}}
										  	 }


%------------------------------------------------------------------------------------------------------------
% image
%------------------------------------------------------------------------------------------------------------

\newcount \filextensionnum
\filextensionnum = 0
\newcount \correctionenonce
\correctionenonce = 1

			\filextensionnum=2 \correctionenonce=1 


%\ifnum \filextensionnum = 1
%\newcommand{\filext}[1]{#1.eps}
%\newcommand{\filefig}[1]{\input{#1.tex}}
%\else
%\newcommand{\filext}[1]{#1.pdf}
%\newcommand{\filefig}[1]{\includegraphics{#1.pdf}}
%\fi

%------------------------------------------------------------------------------------------------------------
% test pour savoir si c'est le premier passage
%------------------------------------------------------------------------------------------------------------
\newcommand{\firstpassagedo}[1]{ \ifnum\nbpassages=1 #1 \fi }


%-----------------------------------------------------------------------------------------------------------
% document 
%-----------------------------------------------------------------------------------------------------------
\begin{document}

\begin{tabular}{l} \tiny  .  \\ \tiny . \normalsize


\else

\ifnum\nbpassages=4
\nbpassages=5
\fi 

\ifnum\nbpassages=3
\nbpassages=4
\fi 

\ifnum\nbpassages=2
\nbpassages=3
\fi 

\ifnum\nbpassages=1
\nbpassages=2
\fi 



\fi


\unitlength 1mm


\begin{picture}(170,95)(10,-20)

%--------------------------------------------------

\put(10,35)		{\framebox(30,5){
									\begin{tabular}{c}structurel\end{tabular}
							}}
									
\put(10,10)		{\framebox(30,20){
									\begin{tabular}{l}
									code de Freeman \\
									polygone \\
									splines \\
									invariants
									\end{tabular}
							}}

%--------------------------------------------------

\put(45,35)		{\framebox(40,5){
									\begin{tabular}{c}global\end{tabular}
							}}
									
\put(45,-15)		{\framebox(40,45){
									\begin{tabular}{l}
									p�rim�tre				\\
									compacit�				\\
									excentricit�	\\
									signature d'une forme		\\
									distance de Hausdorff		\\
									descripteurs de Fourier	\\
									variation d'�chelle				\\
									auto-r�gressif					\\
									elastic matching
									\end{tabular}
							}}
									
%--------------------------------------------------

\put(90,35)		{\framebox(48,5){
									\begin{tabular}{c}global\end{tabular}
							}}
									
\put(90,-15)		{\framebox(48,45){
									\begin{tabular}{l}
									surface				\\
									nombre d'Euler				\\
									excentricit�\\
									moments g�om�triques\\
									moments de Zernike\\
									moments de Legendre\\
									descripteurs de \\ \quad Fourier g�n�riques\\
									m�thode grille	\\
									matrices de formes
									\end{tabular}
							}}
									
%--------------------------------------------------

\put(140,35)		{\framebox(40,5){
									\begin{tabular}{c}structurel\end{tabular}
							}}
									
\put(140,10)		{\framebox(40,20){
									\begin{tabular}{l}
									enveloppe convexe \\
									axe m�dian (squelette) \\
									noyau
									\end{tabular}
							}}

				
%--------------------------------------------------

\put(27,50)		{\framebox(30,5){
									\begin{tabular}{c}contour\end{tabular}
							}}
									
\put(123,50)		{\framebox(30,5){
									\begin{tabular}{c}r�gion\end{tabular}
							}}
									
\put(75,65)		{\framebox(30,5){
									\begin{tabular}{c}forme\end{tabular}
							}}
									

%--------------------------------------------------

\put(90,65)    {\line(0,-1){4}}
\put(42,61)    {\line(1,0){96}}
\put(42,61)    {\vector(0,-1){5}}
\put(138,61)   {\vector(0,-1){5}}

\put(42,50)    {\line(0,-1){4}}
\put(138,50)   {\line(0,-1){4}}
\put(25,46)    {\line(1,0){40}}
\put(113,46)   {\line(1,0){47}}

\put(25,46)   {\vector(0,-1){5}}
\put(65,46)   {\vector(0,-1){5}}
\put(113,46)  {\vector(0,-1){5}}
\put(160,46)  {\vector(0,-1){5}}

\put(25,35)   {\vector(0,-1){5}}
\put(65,35)   {\vector(0,-1){5}}
\put(113,35)  {\vector(0,-1){5}}
\put(160,35)  {\vector(0,-1){5}}

\end{picture}


%-----------------------------------------------------------------------------------------------------
% afin d'�viter d'inclure plusieurs ce fichier
%-----------------------------------------------------------------------------------------------------

\ifnum\nbpassages=1

 \tiny . \end{tabular}

%-----------------------------------------------------------------------------------------------------
\end{document}
%-----------------------------------------------------------------------------------------------------


\else


\ifnum\nbpassages=2
\nbpassages=1
\fi 

\ifnum\nbpassages=3
\nbpassages=2
\fi 

\ifnum\nbpassages=4
\nbpassages=3
\fi 

\ifnum\nbpassages=5
\nbpassages=4
\fi 


\fi

